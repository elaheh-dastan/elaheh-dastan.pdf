\section{Experience}

\cventry{2021 -- present}{Senior ML Engineer}{Snapp!}{Tehran, Iran}{}{
      \begin{itemize}[label=\textbullet]
            \item Conceptualized \textbf{matrix factorization} techniques like \textbf{ALS} to recommend speed for streets that don't have sufficient data which \textbf{improved coverage from one million shard streets to 3 millions}.
            \item Started a project to give ETA(estimated time of arrival) using ML models (\textbf{Random Forests} and \textbf{Fully Connected Neural Networks}) in over 5 cities in both Iran and Iraq that overally improved R2 by 20\%.
            \item Developed a complete pipeline to train, version and deploy models using Spark, MLFlow, Katib, Feast, TensorflowServer, FastAPI that reduced process time 70\%.
            \item Load tested models using \textbf{K6}. The NN model's p90 response time was around 10ms and Random Forest's p90 response time was around 30ms. Made sure of online benchmarking, monitoring, tracing.
            \item Clustered Iran cities \textbf{Hierarchically} from over 40 down to 10 which helped in reducing number of models.
      \end{itemize}
}

\vspace{.5cm}

\cventry{2020 -- 2021}{Senior ML Engineer}{Snapp!}{Tehran, Iran}{}{
      \vspace{.4cm}
      \begin{itemize}[label=\textbullet]
            \item Re-designed data pipeline and services to improve its performance. Replaced old \textbf{spark}-based solution for driver location gathering with Golang to handle 40K driver
                  locations per second instead of former 8K per second.
            \item Upgraded \textbf{Cassandra} cluster to handle 200k per second write ops instead of 73k per second
            \item Used \textbf{Apache beam} over \textbf{Spark} so we could have tests for our pipeline stages.
            \item Changed the structure of data gathering to data driven using \textbf{Kafka} as CMQ that handles over 80k messages per second
            \item Deployed data tools in data pipeline for example \textbf{Airflow} for data gathering and preprocessing, \textbf{AutoML} tools like \textbf{H2O} to reduce time in training and testing models, \textbf{Feast} as feature store etc
            \item Mentored interns and onboard them on projects. Helped team on interviews and hiring process
            \item used \textbf{ONNX} to increase inference speed by 10\%
            \item Contributed in team's 2022 3rd IEEE Intelligent Vehicles Symposium (IV)
      \end{itemize}
}

\vspace{0.5cm}

\cventry{2018 -- 2020}{ML Engineer}{Avidnet Technology}{Tehran, Iran}{}{
      \vspace{.4cm}
      \begin{itemize}[label=\textbullet]
            \item Used \textbf{Kalman filtering} to know where a person is.
            \item Predicted where the person should be in a time bucket and alert otherwise.
            \item Used \textbf{Tensorflow Lite} to run the model on mobile phones.
            \item Set up a \textbf{Kafka} pipeline to gather data from sensors using \textbf{Protobuf} and then stores them into our \textbf{DataLake} (which is set up using \textbf{Postgres}).
            \item Drew the map of the house using accumulated GPS points
            \item Made video calls automaticallywith emergency contacts in case of abnormal behavior
            \item Provided in-application video call solution using WebRTC based on Pion Framework in Golang
            \item Set up turn server on AWS and use ELBs to open UDP/TCP ports
            \item Handled 1K concurrent calls based on our distributed design
            \item Set up project on Google Cloud Platform (\textbf{GCP}) Compute Engine using \textbf{Terraform}.
      \end{itemize}
}
